\documentclass[11pt,letterpaper]{article}
\usepackage[top=3cm, bottom=2cm, left=2cm, right=2cm, columnsep=20pt]{geometry}
\usepackage{pdfpages}
\usepackage{graphicx}
\usepackage{etoolbox}
\apptocmd{\sloppy}{\hbadness 10000\relax}{}{}
% \usepackage[numbers]{natbib}
\usepackage[T1]{fontenc}
\usepackage{ragged2e}
\usepackage[french]{babel}
\usepackage{listings}
\usepackage{color}
\usepackage{soul}
\usepackage[utf8]{inputenc}
\usepackage[export]{adjustbox}
\usepackage{caption}
\usepackage{amsmath}
\usepackage{amssymb}
\usepackage{float}
\usepackage{csquotes}
\usepackage{fancyhdr}
\usepackage{wallpaper}
\usepackage{siunitx}
\usepackage[indent]{parskip}
\usepackage{textcomp}
\usepackage{gensymb}
\usepackage{multirow}
\usepackage[hidelinks]{hyperref}
\usepackage{abstract}
\renewcommand{\abstractnamefont}{\normalfont\bfseries}
\renewcommand{\abstracttextfont}{\normalfont\itshape}
\usepackage{titlesec}
\titleformat{\section}{\large\bfseries}{\thesection}{1em}{}
\titleformat{\subsection}{\normalsize\bfseries}{\thesubsection}{1em}{}
\titleformat{\subsubsection}{\normalsize\bfseries}{\thesubsubsection}{1em}{}

\usepackage{xcolor}
\definecolor{codegreen}{rgb}{0,0.6,0}
\definecolor{codegray}{rgb}{0.5,0.5,0.5}
\definecolor{codepurple}{rgb}{0.58,0,0.82}
\definecolor{backcolour}{rgb}{0.95,0.95,0.92}
\lstdefinestyle{mystyle}{
    backgroundcolor=\color{backcolour},   
    commentstyle=\color{codegreen},
    keywordstyle=\color{magenta},
    numberstyle=\tiny\color{codegray},
    stringstyle=\color{codepurple},
    basicstyle=\ttfamily\footnotesize,
    breakatwhitespace=false,         
    breaklines=true,                 
    captionpos=b,                    
    keepspaces=true,                 
    numbers=left,                    
    numbersep=5pt,                  
    showspaces=false,                
    showstringspaces=false,
    showtabs=false,                  
    tabsize=2
}
\lstset{style=mystyle}

\usepackage[most]{tcolorbox}
\newtcolorbox{note}[1][]{
  enhanced jigsaw,
  borderline west={2pt}{0pt}{black},
  sharp corners,
  boxrule=0pt, 
  fonttitle={\large\bfseries},
  coltitle={black},
  title={Note:\ },
  attach title to upper,
  #1
}

%----------------------------------------------------

\setlength{\parindent}{0pt}
\DeclareCaptionLabelFormat{mycaptionlabel}{#1 #2}
\captionsetup[figure]{labelsep=colon}
\captionsetup{labelformat=mycaptionlabel}
\captionsetup[figure]{name={Figure }}
\newcommand{\inlinecode}{\normalfont\texttt}
\usepackage{enumitem}
\setlist[itemize]{label=\textbullet}

\begin{document}
\begin{titlepage}
\center

\begin{figure}
    \ThisULCornerWallPaper{.4}{Polytechnique_signature-RGB-gauche_FR.png}
\end{figure}
\vspace*{2 cm}

\textsc{\Large \textbf{PHS2223 --} Introduction à l'optique moderne}\\[0.5cm]
\large{\textbf{Équipe : 04}}\\[1.5cm]

\rule{\linewidth}{0.5mm} \\[0.5cm]
\Large{\textbf{Expérience 2}} \\[0.2cm]
\text{Objectif de caméra}\\
\rule{\linewidth}{0.2mm} \\[2.3cm]

\large{\textbf{Présenté à}\\
  Guillaume Sheehy\\
  Esmat Zamani\\[2.5cm]
  \textbf{Par :}\\
  Émile \textbf{Guertin-Picard} (2208363)\\
  Laura-Li \textbf{Gilbert} (2204234)\\
  Tom \textbf{Dessauvages} (2133573)\\[3cm]}

\large{\today\\
Département de Génie Physique\\
Polytechnique Montréal\\}

\end{titlepage}

%----------------------------------------------------

\tableofcontents
\pagenumbering{roman}
\newpage

\pagestyle{fancy}
\setlength{\headheight}{14pt}
\renewcommand{\headrulewidth}{0pt}
\fancyfoot[R]{\thepage}

\pagestyle{fancy}
\fancyhf{}
\renewcommand{\headrulewidth}{1pt}
\fancyhead[L]{\textbf{PHS2223}}
\fancyhead[C]{Rapport préliminaire}
\fancyhead[R]{\today}
\fancyfoot[R]{\thepage}

\pagenumbering{arabic}
\setcounter{page}{1}

%----------------------------------------------------

\section{Introduction}

\section{Théorie}

\subsection{Système de caméra}

La figure \ref{schema_syst}, tirée de l'énoncé du laboratoire \textcolor{red}{Source A}, présente
l'intérieur du système de caméra sous forme de schéma des lentilles :

\begin{figure}[H]
  \centering
  \includegraphics[scale=0.1]{systeme_optique.png}
  \caption{Schéma du système optique d'objectif de caméra à zoom et focus ajustable}
  \label{schema_syst}
\end{figure}

% Source A : énoncé du lab

\subsubsection{Grossissement}

\subsubsection{Profondeur de champ}

\subsection{Modélisation mathématique par matrice de transfert}

\section{Méthodologie}

\subsection{Montage}

%TODO : présenter ET expliquer le montage

% info du pixel pitch AKA taille d'un pixel (fiche technique de la caméra)
% https://www.thorlabs.com/catalogpages/obsolete/2020/DCC1645C.pdf
% est cité plus tard dans l'hypothèse

\subsection{Liste des images à prendre}

\section{Hypothèses}

% TODO : code en annexe

Afin de pouvoir prédire les phénomènes qui pourront être observés en laboratoire, un
programme Python disponible en annexe a été développé afin de résoudre analytiquement
le système par la méthode des matrices. Ce programme commence tout d'abord par construire
la matrice de transfert du système, puis il paramétrise l'écart entre les lentilles selon
la position de $L_2$ par rapport à $L_1$. Enfin, il calcule les différentes caractéristiques
qui suivent.

\subsection{Profondeur de champ}

Premièrement, le programme calcule la profondeur de champ en fonction de la position de
la lentille $L_2$. Le résultat est présenté à la figure \ref{prof_champ_plot}.

\begin{figure}[H]
  \centering
  \includegraphics[scale=0.55]{prof_champ.png}
  \caption{Graphique de la profondeur de champ en fonction de la position de la lentille 
  $L_2$ suite au calcul de Python}
  \label{prof_champ_plot}
\end{figure}

Ainsi, il est possible de faire comme hypothèse que la profondeur du champ n'est pas affectée
par la position de $L_2$.

\subsection{Résolution}

Ensuite, la figure \ref{gross_plot} montre le grossissement de l'image en fonction de la
position de $L_2$.

\begin{figure}[H]
  \centering
  \includegraphics[scale=0.55]{grossissement.png}
  \caption{Graphique du grossissement d'une image en fonction de la position de la lentille
  $L_2$ suite au calcul de Python.}
  \label{gross_plot}
\end{figure}

Découlant directement de ce résultat, il est possible de visualiser la résolution en fonction
de la position de la lentille, tel que montré à la figure \ref{res_plot}. Cela montre la taille
d'un pixel de la caméra (\textit{pixel pitch}), qui vaut 3.6 micromètres pour la caméra THORLABS, dans le plan objet \textcolor{red}{Source B}.

% Source B : https://www.thorlabs.com/catalogpages/obsolete/2020/DCC1645C.pdf

\begin{figure}[H]
  \centering
  \includegraphics[scale=0.55]{Resolution.png}
  \caption{Graphique de la résolution de la caméra en fonction de la position de la lentille
  $L_2$ suite au calcul de Python.}
  \label{res_plot}
\end{figure}

Ainsi, il est possible de prédire que le grossissement (qui est inversé en raison du signe 
négatif) augmente plus la distance entre $L_2$ et $L_1$ augmente. Sans surprise, une relation 
réciproque est prédite pour la résolution, qui décroît plus cette distance augmente.

\subsection{Facteur de zoom}

Enfin, selon la figure \ref{gross_plot}, il est possible de prendre les deux extrêmes de la courbe 
pour trouver une prédiction dufacteur de zoom. Calculé par le programme, le résultat est un facteur
d'environ 0.45.
\clearpage

% \bibliographystyle{unsrtnat}
% \bibliography{My_Library}

\end{document}
